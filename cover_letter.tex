\documentclass[a4paper,12pt]{article} 
\usepackage{times}
\usepackage[top=1in,bottom=1in,left=1in,right=1in,nohead]{geometry}
\usepackage{graphics}

\begin{document}
\renewcommand{\rmdefault}{ptm}
\pagestyle{empty} 

\setlength{\parindent}{0mm} 
\setlength{\parskip}{5mm}

\begin{flushright}
\today
\end{flushright}

\emph{To:}\newline
Dr. Damian Pattinson\newline
Editorial Director\newline
PLOS ONE

\emph{From:}\newline
Dr. Marcus W. Beck\newline
ORISE Research Participation Program\newline
US Environmental Protection Agency\newline
beck.marcus@epa.gov, 850-934-2480\vspace{0.1in}

Enclosed please find my manuscript, entitled `SWMPr: An R Package for Retrieving, Organizing, and Analyzing Environmental Data for Estuaries', to be considered as a research article in PLOS ONE. A continuing challenge in the environmental sciences is the effective use of continuous monitoring data.  The increasing quantity of data from automated sensor networks suggests that more appropriate methods to synthesize and interpret this information are needed. This article describes the SWMPr (`swamper') software package that was developed to address research needs of the US National Estuarine Research Reserve System (NERRS) using the open-source statistical programming language R.  The System-Wide Monitoring Program (SWMP) was implemented by NERRS in 1995 to collect continuous water quality data at over 140 fixed stations at each of the estuaries in the reserve system.  The SWMPr package address many common issues working with large datasets from automated sensor networks, such as data pre-processing, combining data from different sources, and exploratory analyses to identify parameters of interest. Additionally, a cross-reserve comparison of water quality trends and ecosystem metabolism estimates is provided to illustrate the utility of the package. 

The manuscript has not been previously submitted to PLOS ONE and I have had no prior interactions with PLOS. Please note that Dr. Jeffrey Hollister (US EPA) has already provided a review of an earlier draft. I greatly appreciate the opportunity to publish my work in PLOS ONE.  Below is a suggested list of appropriate editors to handle my manuscript.

Zaid Abdo,
Institution and Department: Agricultural Research Service,
UNITED STATES

Ruth H. Carmichael,
Dauphin Island Sea Lab; University of South Alabama,
UNITED STATES

Juan Carlos Molinero,
GEOMAR: Helmholtz Center for Ocean Research,
GERMANY

Tomoya Iwata,
University of Yamanashi,
JAPAN

\vspace{0.1in}

\hspace{4.5in}Respectfully,

\hspace{4.5in}Dr. Marcus W. Beck

\end{document}