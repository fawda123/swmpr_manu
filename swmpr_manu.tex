% Template for PLoS
% Version 3.1 February 2015
%
% % % % % % % % % % % % % % % % % % % % % % % 
%
% Once your paper is accepted for publication, 
% PLEASE REMOVE ALL TRACKED CHANGES in this file and leave only
% the final text of your manuscript.
%
% There are no restrictions on package use within the LaTeX files except that 
% no packages listed in the template may be deleted.
%
% Please do not include colors or graphics in the text.
%
% Please do not create a heading level below \subsection. For 3rd level headings, use \paragraph{}.
%
% % % % % % % % % % % % % % % % % % % % % % %
%
% -- FIGURES AND TABLES
%
% Please include tables/figure captions directly after the paragraph where they are first cited in the text.
%
% DO NOT INCLUDE GRAPHICS IN YOUR MANUSCRIPT
% - Figures should be uploaded separately from your manuscript file. 
% - Figures generated using LaTeX should be extracted and removed from the PDF before submission. 
% - Figures containing multiple panels/subfigures must be combined into one image file before submission.
% For figure citations, please use "Fig." instead of "Figure".
% See http://www.plosone.org/static/figureGuidelines for PLOS figure guidelines.
%
% Tables should be cell-based and may not contain:
% - tabs/spacing/line breaks within cells to alter layout or alignment
% - vertically-merged cells (no tabular environments within tabular environments, do not use \multirow)
% - colors, shading, or graphic objects
% See http://www.plosone.org/static/figureGuidelines#tables for table guidelines.
%
% For tables that exceed the width of the text column, use the adjustwidth environment as illustrated in the example table in text below.
%
% % % % % % % % % % % % % % % % % % % % % % % %
%
% -- EQUATIONS, MATH SYMBOLS, SUBSCRIPTS, AND SUPERSCRIPTS
%
% IMPORTANT
% Below are a few tips to help format your equations and other special characters according to our specifications. For more tips to help reduce the possibility of formatting errors during conversion, please see our LaTeX guidelines at http://www.plosone.org/static/latexGuidelines
%
% Please be sure to include all portions of an equation in the math environment.
%
% Do not include text that is not math in the math environment. For example, CO2 will be CO\textsubscript{2}.
%
% Please add line breaks to long display equations when possible in order to fit size of the column. 
%
% For inline equations, please do not include punctuation (commas, etc) within the math environment unless this is part of the equation.
%
% % % % % % % % % % % % % % % % % % % % % % % %

\documentclass[10pt,letterpaper]{article}\usepackage[]{graphicx}\usepackage[]{color}
%% maxwidth is the original width if it is less than linewidth
%% otherwise use linewidth (to make sure the graphics do not exceed the margin)
\makeatletter
\def\maxwidth{ %
  \ifdim\Gin@nat@width>\linewidth
    \linewidth
  \else
    \Gin@nat@width
  \fi
}
\makeatother

\definecolor{fgcolor}{rgb}{0.345, 0.345, 0.345}
\newcommand{\hlnum}[1]{\textcolor[rgb]{0.686,0.059,0.569}{#1}}%
\newcommand{\hlstr}[1]{\textcolor[rgb]{0.192,0.494,0.8}{#1}}%
\newcommand{\hlcom}[1]{\textcolor[rgb]{0.678,0.584,0.686}{\textit{#1}}}%
\newcommand{\hlopt}[1]{\textcolor[rgb]{0,0,0}{#1}}%
\newcommand{\hlstd}[1]{\textcolor[rgb]{0.345,0.345,0.345}{#1}}%
\newcommand{\hlkwa}[1]{\textcolor[rgb]{0.161,0.373,0.58}{\textbf{#1}}}%
\newcommand{\hlkwb}[1]{\textcolor[rgb]{0.69,0.353,0.396}{#1}}%
\newcommand{\hlkwc}[1]{\textcolor[rgb]{0.333,0.667,0.333}{#1}}%
\newcommand{\hlkwd}[1]{\textcolor[rgb]{0.737,0.353,0.396}{\textbf{#1}}}%

\usepackage{framed}
\makeatletter
\newenvironment{kframe}{%
 \def\at@end@of@kframe{}%
 \ifinner\ifhmode%
  \def\at@end@of@kframe{\end{minipage}}%
  \begin{minipage}{\columnwidth}%
 \fi\fi%
 \def\FrameCommand##1{\hskip\@totalleftmargin \hskip-\fboxsep
 \colorbox{shadecolor}{##1}\hskip-\fboxsep
     % There is no \\@totalrightmargin, so:
     \hskip-\linewidth \hskip-\@totalleftmargin \hskip\columnwidth}%
 \MakeFramed {\advance\hsize-\width
   \@totalleftmargin\z@ \linewidth\hsize
   \@setminipage}}%
 {\par\unskip\endMakeFramed%
 \at@end@of@kframe}
\makeatother

\definecolor{shadecolor}{rgb}{.97, .97, .97}
\definecolor{messagecolor}{rgb}{0, 0, 0}
\definecolor{warningcolor}{rgb}{1, 0, 1}
\definecolor{errorcolor}{rgb}{1, 0, 0}
\newenvironment{knitrout}{}{} % an empty environment to be redefined in TeX

\usepackage{alltt}
\usepackage[top=0.85in,left=2.75in,footskip=0.75in]{geometry}

% Use adjustwidth environment to exceed column width (see example table in text)
\usepackage{changepage}

% Use Unicode characters when possible
\usepackage[utf8]{inputenc}

% textcomp package and marvosym package for additional characters
\usepackage{textcomp,marvosym}

% fixltx2e package for \textsubscript
\usepackage{fixltx2e}

% amsmath and amssymb packages, useful for mathematical formulas and symbols
\usepackage{amsmath,amssymb}

% cite package, to clean up citations in the main text. Do not remove.
\usepackage{cite}

% Use nameref to cite supporting information files (see Supporting Information section for more info)
\usepackage{nameref}

% line numbers
\usepackage[right]{lineno}

% ligatures disabled
\usepackage{microtype}
\DisableLigatures[f]{encoding = *, family = * }

% rotating package for sideways tables
\usepackage{rotating}

% packages I added
\usepackage{cleveref}
\usepackage[colorlinks=true,allcolors=Blue]{hyperref}
\usepackage[usenames,dvipsnames]{xcolor}

% Remove comment for double spacing
%\usepackage{setspace} 
%\doublespacing

% Text layout
\raggedright
\setlength{\parindent}{0.5cm}
\textwidth 5.25in 
\textheight 8.75in

% Bold the 'Figure #' in the caption and separate it from the title/caption with a period
% Captions will be left justified
\usepackage[aboveskip=1pt,labelfont=bf,labelsep=period,justification=raggedright,singlelinecheck=off]{caption}

% Use the PLoS provided BiBTeX style
\bibliographystyle{plos2015}

% Remove brackets from numbering in List of References
\makeatletter
\renewcommand{\@biblabel}[1]{\quad#1.}
\makeatother

% Leave date blank
\date{}

% Header and Footer with logo
\usepackage{lastpage,fancyhdr,graphicx}
\usepackage{epstopdf}
\pagestyle{myheadings}
\pagestyle{fancy}
\fancyhf{}
\lhead{\includegraphics[width=2.0in]{PLOS-submission.eps}}
\rfoot{\thepage/\pageref{LastPage}}
\renewcommand{\footrule}{\hrule height 2pt \vspace{2mm}}
\fancyheadoffset[L]{2.25in}
\fancyfootoffset[L]{2.25in}
\lfoot{\sf PLOS}

% for acronyms
\usepackage{acronym}

%% Include all macros below

\newcommand{\lorem}{{\bf LOREM}}
\newcommand{\ipsum}{{\bf IPSUM}}

% cleveref options
\crefname{table}{Table}{Tables}
\crefname{figure}{Fig.}{Figs.}
\renewcommand{\figurename}{Fig.}

% acronyms
\acrodef{CDMO}{Centralized Data Management Office}
\acrodef{DO}{dissolved oxygen}
\acrodef{NERRS}{National Estuarine Research Reserve System}
\acrodef{QAQC}{quality assurance/quality control}
\acrodef{SWMP}{System Wide Monitoring Program}

%% END MACROS SECTION

%knitr options


\IfFileExists{upquote.sty}{\usepackage{upquote}}{}
\begin{document}
\vspace*{0.35in}

% Title must be 250 characters or less.
% Please capitalize all terms in the title except conjunctions, prepositions, and articles.
\begin{flushleft}
{\Large
\textbf\newline{SWMPr: An R package for retrieving, organizing, and analyzing environmental data for estuaries}
}
\newline
% Insert author names, affiliations and corresponding author email (do not include titles, positions, or degrees).
\\
Marcus William Beck\textsuperscript{1, *}

\bigskip
\bf{1} ORISE Research Participation Program, USEPA National Health and Environmental Effects Research Laboratory, Gulf Ecology Division, 1 Sabine Island Drive, Gulf Breeze, FL 32651, USA
\\
\bigskip

* beck.marcus@epa.gov

\end{flushleft}
% Please keep the abstract below 300 words
\section*{Abstract}
Standardized monitoring programs have vastly improved the quantity and quality of data that form the basis of environmental decision-making.  One example in the United States is the \ac{SWMP} that was implemented in 1995 by the federally-funded \ac{NERRS}.  This program has provided two decades of continuous monitoring data at over 300 fixed stations in 28 estuaries across the United States.  \ac{SWMP} data have been used in a variety applications with the general objective of describing dynamics of estuarine ecosystems to better inform effective coastal management.  However, simple tools for processing and evaluating the large and increasing quantity of data provided by the monitoring network have prevented large-scale comparisons between systems and, in some cases, simple trend analysis of water quality parameters at individual sites.  We describe a new open-source software package, SWMPr, developed in program R for use with \ac{SWMP} environmental data.  The package provides several functions that facilitate data retrieval, organization, and analysis of time series data to describe water quality, weather, and nutrient dynamics in the reserve estuaries.  Previously unavailable functions for estuaries are also provided to estimate rates of ecosystem metabolism using the open-water method.  Tools included with the SWMPr package have facilitated a cross-reserve comparison of trends, including simple evaluation of changes over time and comparisons of patterns in primary productivity.  Overall, the package provides an effective approach to link quantitative information with analysis tools that will greatly inform management programs aimed at coastal protection and restoration.

\linenumbers

\section*{Introduction}

The development of low-cost, automated sensors that have the ability to collect data in near real-time has enabled a proliferation of standardized environmental monitoring programs \cite{Glasgow04,Fries08}.  These programs provide access to invaluable sources of data that can be used to address a variety of research and management objectives.  Applications from automated remote sensors are numerous with notable examples including prediction of harmful algal blooms and toxicants in aquatic systems \cite{Reed10}, development of a hydrometeorological monitoring network to support flash flood warning programs \cite{HADS15}, and automated detection of airborne chemical warfare agents \cite{Sanders01}.  Automated remote monitoring programs offer several advantages over traditional site-specific, field-based methods including streamlining of data acquisition, minimizing human error, and reducing the overall cost of the collection process \cite{Glasgow04}.  However, the growing quantity of available information to address relevant questions has contributed to the notion of `big data' science where analysis limitations are more often defined by compututational requirements and signal identification in the presence of noise rather than the availability of information.  Growing concerns over the use of adequate quality assurance and control methods, challenges for synthesis and interpretation, and increased emphasis on exploratory-based analytical techniques have characterized applications of data from automated monitoring programs \cite{Campbell13,Millie13}.

The \acl{NERRS} (\acs{NERRS}, \url{http://www.nerrs.noaa.gov/}) is a network of 28 estuarine reserves in the United States that was established by the Coastal Zone Management Act of 1972.  The reserves represent different biogeographic regions and estuarine types that were chosen to address multiple goals for long-term research, monitoring, education, and stewardship in support of coastal management.  As part of this effort, the \acf{SWMP} program was implemented in 1995 at over 300 stations at each of the reserves to provide a robust, long-term monitoring system for water quality, weather, and land-use/habitat change.  The \ac{SWMP} network has provided a continous source of data collected at near real-time at each of the reserves with the intent to evaluate natural and anthrogogenic causes of spatiotemporal variation in environmental condition and ecosystem function.  These data have been useful for evaluating relevant characteristics at individual reserves (eg., \cite{Bulthius95,Dix08}) and differences between reserves (e.g., ecosystem metabolism \cite{Caffrey03,Caffrey04}, tidal characteristics \cite{Sanger02}, dissolved oxygen \cite{Wenner04}).  However, no cross-reserve comparisons have been conducted within the last decade despite the online availability of current \ac{SWMP} data.  \ac{NERRS} researchers and staff have also expressed a need for quantitative analysis tools to evaluate trends in water quality time series given the quantity of data provided by \ac{SWMP} \cite{SWMP14}.          

This article describes a software package that was developed to address research needs of the \ac{NERRS} program using the open-source statistical programming language R \cite{RDCT14}.  SWMPr (pronounced `swamper') is an R package that contains functions for retrieving, organizing, and analyzing estuary monitoring data from the System Wide Monitoring Program (\href{http://nerrs.noaa.gov/RCDefault.aspx?ID=18}{SWMP}).  Functions provided by SWMPr address many of the common issues working with large datasets created from automated sensor networks, such as data pre-processing to remove unwanted information, combining data from different sources, and exploratory analyses to identify key parameters of interest.  Additionally, a cross-reserve comparison of current ecosystem metabolism estimates is provided to illustrate a potential application using the functions in this package.  The software is provided specifically for use with \ac{NERRS} data, although many of the applications are relevant addressing common challenges working with large datasets.

\section*{SWMP overview and data retrieval}

Four core data elements are collected through the \ac{SWMP} monitoring network: abiotic monitoring data, biotic observations, habitat and land use mapping, and sentinel monitoring.  The SWMPr package is developed for the continous abiotic monitoring network which includes a majority of the data collected by \ac{SWMP}.  Abiotic elements monitored at each reserve include water quality (water temperature, specific conductivity, salinity, dissolved oxygen concentration, dissolved oxygen saturation, depth, pH, turbidity, chlorophyll fluorescence), weather (air temperature, relative humidity, barometric pressure, wind speed, wind direction, photosynthetically active radiation, precipitation), and nutrient data (orthophosphate, ammonium, nitrite, nitrate, nitrite + nitrate, chlorophyll a).  Each reserve has no less than four water quality stations and one weather station at fixed locations.  Water quality and weather data are collected at 15 minute intervals, whereas nutrient data are collected monthly at each water quality station.  All data are made accessible through the \ac{CDMO} web portal, where multiple \ac{QAQC} measures are used to screen the information.  The final data include all observations with relevant \ac{QAQC} flags indicating the appropriate qualifier (\href{http://cdmo.baruch.sc.edu/data/qaqc.cfm}{view codes}).

The \ac{CDMO} web portal was established to support priority areas of \ac{SWMP} that focus on the continuation and advancement of data and information management.  As such, \ac{CDMO} provides access to over 35 million water quality, weather, and nutrient records that have been authenticated through systematic \ac{QAQC} procedures.  Estuary data must be obtained from \ac{CDMO} prior to using most of the functions within the SWMPr package.  In most cases, the analysis needs will typically define the location, date range, and parameters of interest that need to be obtained from \ac{CDMO}.  All stations in the \ac{SWMP} network are identified by a 7 or 8 character name that specifies the reserve, station, and parameter type.  For example, `apaebwq' is the water quality identifier for the East Bay station at the Apalachicola reserve.  Similarly, a suffix of `met' or `nut' would specify the weather (meteorological) or nutrients station.  

Prior to any data request to the \ac{CDMO}, the site, parameter type, and date ranges need to be identified based on the analysis needs.  All reserve names, stations, and date ranges for the water quality, weather, or nutrients data can be viewed on the \ac{CDMO} \href{http://cdmo.baruch.sc.edu/}{website}. Alternatively, the \texttt{site\_codes} (all sites) or \texttt{site\_codes\_ind} (single site) functions provided by the SWMPr package can be used to view the same information.  As noted below, the computer's IP address must be registered by \ac{CDMO} staff before using the data retrieval functions in SWMPr.  \href{http://cdmo.baruch.sc.edu/webservices.cfm}{Web services} are provided by \ac{CDMO} to provide direct access to \ac{SWMP} data through http requests, in addition to standard graphical user interface options for selecting data.  The data retrieval functions in SWMPr are simple calls to the existing retrieval functions on \ac{CDMO} web services.  For example, the \texttt{all\_params} function in SWMPr uses the \texttt{exportAllParamsXMLNew} function from the web services to retrieve metadata for all the \ac{SWMP} sites.  The text below describes the data retrieval functions in more detail.

\section*{Structure of the SWMPr package}

\subsection*{Installing the package}

The SWMPr package was developed for use with the R statistical programming language and a recent version of R (v3.0.0 or greater) should be installed (see \href{http://cran.r-project.org/}{http://cran.r-project.org/}). The SWMPr package can be installed from \href{https://github.com/fawda123/SWMPr}{GitHub} by executing the following commands at the R terminal.  The package is loaded in the current workspace by using the \texttt{library} command.

\begin{knitrout}
\definecolor{shadecolor}{rgb}{0.969, 0.969, 0.969}\color{fgcolor}\begin{kframe}
\begin{alltt}
\hlkwd{install.packages}\hlstd{(}\hlstr{'devtools'}\hlstd{)}
\hlkwd{library}\hlstd{(devtools)}
\hlkwd{install_github}\hlstd{(}\hlstr{'fawda123/SWMPr'}\hlstd{)}
\hlkwd{library}\hlstd{(SWMPr)}
\end{alltt}
\end{kframe}
\end{knitrout}


The SWMPr package was developed by considering a standard workflow that categorizes the functions as one of three steps based on their intended use: {\bf{\it retrieving}}, {\bf{\it organizing}}, and {\bf{\it analyzing}}.  Functions for retrieving are used to import the data into R as a \texttt{swmpr} object class.  Functions for organizing and analyzing the data provide methods for working with the \texttt{swmpr} object class.  An additional class of functions, termed `miscellaneous', are also included as helpers for the main functions.  The following describes a general approach for using each category of functions based on a standard data workflow.  

\subsection*{Data retrieval}

Three approaches can be used for retrieving \ac{SWMP} data for import into R.  First, functions from the SWMPr package can be used to import the data directly from the online server using \ac{CDMO} web services (\Cref{tab:retrieve}).  To do so, the IP address for the computer making the request must be registered with \ac{CDMO}.  This can be done by following instructions \href{http://cdmo.baruch.sc.edu/webservices.cfm}{here}.  The \texttt{site\_codes} or \texttt{site\_codes\_ind} functions can be used to view the available metadata after a computer is registered with \ac{CDMO}. 

% table for retrieval functions
%latex.default(to_tab[, "Description", drop = F], file = "", caption = "Retrieval functions available from the SWMPr package. Full documentation for each function is in the help file (e.g., execute \\texttt{?all\\_params} at the command line).",     rowlabel = "Function", colheads = "Description", rowname = to_tab$Functions,     caption.loc = "top", col.just = c("p{3.5in}"), label = "tab:retrieve",     table.env = FALSE)%
\begin{table}[!tbp]
\caption{Retrieval functions available from the SWMPr package. Full documentation for each function is in the help file (e.g., execute \texttt{?all\_params} at the command line).\label{tab:retrieve}} 
\begin{center}
\begin{tabular}{lp{3.5in}}
\hline\hline
\multicolumn{1}{l}{Function}&\multicolumn{1}{c}{Description}\tabularnewline
\hline
\texttt{all\_params}&Retrieve up to 100 records starting with the most recent at a given station, all parameters.  Wrapper to \texttt{exportAllParamsXMLNew} function on web services.\tabularnewline
\texttt{all\_params\_dtrng}&Retrieve records of all parameters within a given date range for a station.  Optional argument for a single parameter.  Maximum of 1000 records. Wrapper to \texttt{exportAllParamsDateRangeXMLNew}.\tabularnewline
\texttt{import\_local}&Import files from a local path.  The files must be in a specific format, specifically those returned from the \ac{CDMO} using the zip downloads option for a reserve.\tabularnewline
\texttt{import\_remote}&Import site data from an independent server. These files have been downloaded from \ac{CDMO} up to December 2014, processed using functions in this package, and uploaded to an Amazon server for quicker import into R.\tabularnewline
\texttt{single\_param}&Retrieve up to 100 records for a single parameter starting with the most recent at a given station.  Wrapper to \texttt{exportSingleParamXMLNew} function on web services.\tabularnewline
\texttt{site\_codes}&Metadata for all stations, wrapper to \texttt{exportStationCodesXMLNew} function on web services.\tabularnewline
\texttt{site\_codes\_ind}&Metadata for all stations at a single site, wrapper  to \texttt{NERRFilterStationCodesXMLNew} function on web services.\tabularnewline
\hline
\end{tabular}\end{center}

\end{table}


\begin{knitrout}
\definecolor{shadecolor}{rgb}{0.969, 0.969, 0.969}\color{fgcolor}\begin{kframe}
\begin{alltt}
\hlcom{# retrieve metadata for all sites}
\hlkwd{site_codes}\hlstd{()}

\hlcom{# retrieve metadata for a single site}
\hlkwd{site_codes_ind}\hlstd{(}\hlstr{'apa'}\hlstd{)}
\end{alltt}
\end{kframe}
\end{knitrout}

Due to rate limitations on the \ac{CDMO} server, the retrieval functions return a limited number of records.  The functions are more useful for evaluating short time periods, although these functions could be used iteratively (i.e., with \texttt{for} loops) to obtain longer time series.  Data retrieval functions to access the \ac{CDMO} include \texttt{all\_params}, \texttt{all\_params\_dtrng}, and \texttt{single\_param}.  These are functions that call the existing web protocol methods on the CDMO web services.  \texttt{all\_params} returns the most recent records of all parameters at a station, \texttt{all\_params\_dtrng} returns all records within a date range for all parameters or a single parameter, and \texttt{single\_param} is identical to \texttt{all\_params} except that a single parameter is requested.    

\begin{knitrout}
\definecolor{shadecolor}{rgb}{0.969, 0.969, 0.969}\color{fgcolor}\begin{kframe}
\begin{alltt}
\hlcom{# all parameters for a station, most recent}
\hlkwd{all_params}\hlstd{(}\hlstr{'hudscwq'}\hlstd{)}

\hlcom{# get all parameters within a date range}
\hlkwd{all_params_dtrng}\hlstd{(}\hlstr{'hudscwq'}\hlstd{,} \hlkwd{c}\hlstd{(}\hlstr{'09/10/2012'}\hlstd{,} \hlstr{'02/8/2013'}\hlstd{))}

\hlcom{# get single parameter within a date range}
\hlkwd{all_params_dtrng}\hlstd{(}\hlstr{'hudscwq'}\hlstd{,} \hlkwd{c}\hlstd{(}\hlstr{'09/10/2012'}\hlstd{,} \hlstr{'02/8/2013'}\hlstd{),}
  \hlkwc{param} \hlstd{=} \hlstr{'do_mgl'}\hlstd{)}

\hlcom{# single parameter for a station, most recent}
\hlkwd{single_param}\hlstd{(}\hlstr{'hudscwq'}\hlstd{,} \hlstr{'do_mgl'}\hlstd{)}
\end{alltt}
\end{kframe}
\end{knitrout}

A second approach for importing \ac{SWMP} data into R is using the \texttt{import\_remote} function.  This method imports site data into R from an independent server where \ac{SWMP} data have been downloaded previously and uploaded as \texttt{swmpr} objects in RData binary format.  This approach is advantageous because the files can be imported very quickly and the computer's IP address does not need to be registered.  The files are hosted via third party web services and include almost all available \ac{SWMP} data for each site up to December 2014.  Always use the \ac{CDMO} for the most recent data.  The files can be accessed by copying the URL to a web browser with the site name and file extension (e.g., \href{https://s3.amazonaws.com/swmpalldata/acebbnut.RData}{https://s3.amazonaws.com/swmpalldata/acebbnut.RData}, see the \nameref{supp_info}) or by using the function as follows.

\begin{knitrout}
\definecolor{shadecolor}{rgb}{0.969, 0.969, 0.969}\color{fgcolor}\begin{kframe}
\begin{alltt}
\hlcom{## import a file from an independent server}
\hlkwd{import_remote}\hlstd{(}\hlstr{'acebbwq'}\hlstd{)}
\end{alltt}
\end{kframe}
\end{knitrout}

A third approach is to use the \texttt{import\_local} function to import data into R that are locally available after downloading outside of R (\href{http://cdmo.baruch.sc.edu/get/landing.cfm}{see here}).  This approach may be best for larger, more customized data requests to \ac{CDMO}. Data can be retrieved from the \ac{CDMO} several ways.  The \texttt{import\_local} function is designed for data from the \href{http://cdmo.baruch.sc.edu/aqs/zips.cfm}{zip downloads} feature in the advanced query section of the \ac{CDMO}. The function may also work using data from the \href{http://cdmo.baruch.sc.edu/get/export.cfm}{data export system}, but this feature has not been extensively tested.  The zip downloads feature is an easy way to obtain data from multiple stations in one request.  The downloaded data will be in a compressed folder that includes multiple .csv files by year for a given data type (e.g., apacpwq2002.csv, apacpwq2003.csv, apacpnut2002.csv, etc.).  The \texttt{import\_local} function can be used after the folder is decompressed.

Occasionally, duplicate time stamps are present in the raw data.  The \texttt{import\_local} function handles duplicate entries differently depending on the data type (water quality,  weather, or nutrients).  For water quality and nutrient data, duplicate time stamps are simply removed.  Note that nutrient data often contain replicate samples with similar but not duplicated time stamps within a few minutes of each other.  Replicates with unique time stamps are not removed but can be further processed using \texttt{rem\_reps}.  Weather data prior to 2007 may contain duplicate time stamps at frequencies for 60 (hourly) and 144 (daily) averages, in addition to 15 minute frequencies.  Duplicate values that correspond to the smallest value in the frequency column (15 minutes) are retained.  

\begin{knitrout}
\definecolor{shadecolor}{rgb}{0.969, 0.969, 0.969}\color{fgcolor}\begin{kframe}
\begin{alltt}
\hlcom{# import data for apaebmet that you downloaded}

\hlcom{# this is an example path with the csv files, change as needed}
\hlstd{path} \hlkwb{<-} \hlstr{'C:/my_path/'}

\hlcom{# import, do not include file extension}
\hlkwd{import_local}\hlstd{(path,} \hlstr{'apaebmet'}\hlstd{)}
\end{alltt}
\end{kframe}
\end{knitrout}

All data retrieval functions return a \texttt{swmpr} object that includes relevant data and several attributes describing the dataset.  The data include a \texttt{datetimestamp} column in the appropriate timezone for a station.  Note that the \texttt{datetimestamp} is standard time for each timezone and does not include daylight savings. Additional columns include parameters for a given data type (weather, nutrients, or water quality) and correspondingg \ac{QAQC} columns if returned from the initial data request.  The attributes for a \texttt{swmpr} object include \texttt{names} of the dataset, \texttt{row.names} of the dataset, \texttt{class} (character string indicating \texttt{swmpr} and \texttt{data.frame}) \texttt{station} (7 or 8 characters identifying the station),  \texttt{parameters} (character vector of data columns, e.g., \texttt{`do\_mgl'}), \texttt{qaqc\_cols} (logical \texttt{T} or \texttt{F} if present or not), \texttt{date\_rng} (\texttt{POSIXct} vector of minimum/maximum dates), \texttt{timezone} (text string in country/city format), and \texttt{stamp\_class} (class of \texttt{datetimestamp} vector, \texttt{POSIXct} or \texttt{Date}).  Attributes of a \texttt{swmpr} object can be viewed as follows.

\begin{knitrout}
\definecolor{shadecolor}{rgb}{0.969, 0.969, 0.969}\color{fgcolor}\begin{kframe}
\begin{alltt}
\hlcom{# import binary data}
\hlkwd{data}\hlstd{(apadbwq)}
\hlstd{dat} \hlkwb{<-} \hlstd{apadbwq}

\hlcom{# verify that dat is swmpr class}
\hlkwd{class}\hlstd{(dat)}
\end{alltt}
\begin{verbatim}
## [1] "swmpr"      "data.frame"
\end{verbatim}
\begin{alltt}
\hlcom{# all attributes of dat}
\hlkwd{names}\hlstd{(}\hlkwd{attributes}\hlstd{(dat))}
\end{alltt}
\begin{verbatim}
## [1] "names"       "row.names"   "class"       "station"    
## [5] "parameters"  "qaqc_cols"   "date_rng"    "timezone"   
## [9] "stamp_class"
\end{verbatim}
\begin{alltt}
\hlcom{# a single attribute of dat}
\hlkwd{attr}\hlstd{(dat,} \hlstr{'station'}\hlstd{)}
\end{alltt}
\begin{verbatim}
## [1] "apadbwq"
\end{verbatim}
\end{kframe}
\end{knitrout}
 
The \texttt{swmpr} object class was created for use with specific methods following the S3 object definition approach \cite{Wickham14}.  A \texttt{swmpr} object also secondarily inherits methods from the \texttt{data.frame} class, such that common \texttt{data.frame} methods also apply to \texttt{swmpr} objects.  Available methods for the \texttt{swmpr} class are described below and can also be viewed:
 
\begin{knitrout}
\definecolor{shadecolor}{rgb}{0.969, 0.969, 0.969}\color{fgcolor}\begin{kframe}
\begin{alltt}
\hlcom{# available methods for swmpr class}
\hlkwd{methods}\hlstd{(}\hlkwc{class} \hlstd{=} \hlstr{'swmpr'}\hlstd{)}
\end{alltt}
\begin{verbatim}
##  [1] aggregate.swmpr       aggregate_metab.swmpr comb.swmpr           
##  [4] decomp.swmpr          decomp_cj.swmpr       ecometab.swmpr       
##  [7] hist.swmpr            lines.swmpr           na.approx.swmpr      
## [10] plot.swmpr            plot_metab.swmpr      plot_summary.swmpr   
## [13] qaqc.swmpr            qaqcchk.swmpr         rem_reps.swmpr       
## [16] setstep.swmpr         smoother.swmpr        subset.swmpr
\end{verbatim}
\end{kframe}
\end{knitrout}

Example data as raw, comma-separated files have not been included in the package due to size limitations.  However, a sample dataset can be \href{https://s3.amazonaws.com/swmpexdata/zip_ex.zip}{downloaded} for use with the examples below.  This dataset has an identical format as the data returned from the zip downloads feature of the \ac{CDMO}.  Processed verions of the raw data are included with the package as binary data files (RData) to decrease processing times with the examples.  Information for each binary file can be viewed as follows.

\begin{knitrout}
\definecolor{shadecolor}{rgb}{0.969, 0.969, 0.969}\color{fgcolor}\begin{kframe}
\begin{alltt}
\hlcom{# view help files for complementary data}
\hlcom{# all files are samples from Apalachicola Bay}

\hlcom{# cat point station, nutrients}
\hlopt{?}\hlstd{apacpnut}

\hlcom{# cat point station, water quality}
\hlopt{?}\hlstd{apacpwq}

\hlcom{# dry bar station, water quality}
\hlopt{?}\hlstd{apadbwq}

\hlcom{# east bay station, weater}
\hlopt{?}\hlstd{apaebmet}
\end{alltt}
\end{kframe}
\end{knitrout}

\subsection*{Data organizing}

The retrieval functions import the data into R as a swmpr object for use with the organize and analyze functions. The organize functions are used to clean or prepare the data for analysis, including removal of QAQC flags, subsetting, creating a standardized time series vector, and combining data of different types (\Cref{tab:organize}).

The \texttt{qaqc} function is a simple screen to retain values from the data with specified QAQC flags (see the \ac{CDMO} descriptions \href{http://cdmo.baruch.sc.edu/data/qaqc.cfm}{here}).  Each parameter in the \texttt{swmpr} data typically has a corresponding QAQC column of the same name with the added prefix \texttt{f\_}.  Values in the QAQC column specify a flag from -5 to 5.  Generally, only data with the  \texttt{0} \ac{QAQC} flag should be used, which is the default option for the \texttt{QAQC} function.  Data that do not satisfy \ac{QAQC} criteria are converted to \texttt{NA} values.   Additionally, simple filters are used to remove obviously bad values, e.g., wind speed values less than zero or pH values greater than 12. Erroneous data entered as -99 are also removed. Processed data will have QAQC columns removed, in addition to removal of values in the actual parameter columns that do not meet the criteria. 

% table for organize functions
%latex.default(to_tab[, "Description", drop = F], file = "", caption = "Organizing functions available from the SWMPr package. Full documentation for each function is in the help file (e.g., execute \\texttt{?comb.swmpr} at the command line).",     rowlabel = "Function", colheads = "Description", rowname = to_tab$Functions,     caption.loc = "top", col.just = c("p{3.5in}"), label = "tab:organize",     table.env = FALSE)%
\begin{table}[!tbp]
\caption{Organizing functions available from the SWMPr package. Full documentation for each function is in the help file (e.g., execute \texttt{?comb.swmpr} at the command line).\label{tab:organize}} 
\begin{center}
\begin{tabular}{lp{3.5in}}
\hline\hline
\multicolumn{1}{l}{Function}&\multicolumn{1}{c}{Description}\tabularnewline
\hline
\texttt{comb.swmpr}&Combines \texttt{swmpr} objects to a common time series using setstep, such as combining the weather, nutrients, and water quality data for a single station. Only different data types can be combined.\tabularnewline
\texttt{qaqc.swmpr}&Remove \ac{QAQC} columns and remove data based on \ac{QAQC} flag values for a \texttt{swmpr} object.  Only applies if \ac{QAQC} columns are present. \tabularnewline
\texttt{qaqcchk.swmpr}&View a summary of the number of observations in a \texttt{swmpr} object that are assigned to different \ac{QAQC} flags used by \ac{CDMO}.  The output is used to inform further processing but is not used explicitly.\tabularnewline
\texttt{rem\_reps.swmpr}&Remove replicate nutrient data that occur on the same day.  The default is to average replicates.\tabularnewline
\texttt{setstep.swmpr}&Format data from a \texttt{swmpr} object to a continuous time series at a given timestep.  The function is used in \texttt{comb.swmpr} and can also be used with individual stations.\tabularnewline
\texttt{subset.swmpr}&Subset by dates and/or columns for a \texttt{swmpr} object.  This is a method passed to the generic \texttt{subset} function provided in the base package.\tabularnewline
\hline
\end{tabular}\end{center}

\end{table}


\begin{knitrout}
\definecolor{shadecolor}{rgb}{0.969, 0.969, 0.969}\color{fgcolor}\begin{kframe}
\begin{alltt}
\hlcom{# qaqc screen for a swmpr object, retain only '0'}
\hlkwd{qaqc}\hlstd{(dat)}

\hlcom{# retain all data regardless of flag}
\hlkwd{qaqc}\hlstd{(dat,} \hlkwc{qaqc_keep} \hlstd{=} \hlkwa{NULL}\hlstd{)}

\hlcom{# retain only '0' and '-1' flags}
\hlkwd{qaqc}\hlstd{(dat,} \hlkwc{qaqc_keep} \hlstd{=} \hlkwd{c}\hlstd{(}\hlnum{0}\hlstd{,} \hlopt{-}\hlnum{1}\hlstd{))}
\end{alltt}
\end{kframe}
\end{knitrout}

Viewing the number of observations for each parameter that are assigned to a \ac{QAQC} flag may be useful for deciding how to process the data with \texttt{qaqc}.  The \texttt{qaqcchk} function can be used to view this information.

\begin{knitrout}
\definecolor{shadecolor}{rgb}{0.969, 0.969, 0.969}\color{fgcolor}\begin{kframe}
\begin{alltt}
\hlcom{# view the number of observations in each QAQC flag}
\hlkwd{qaqcchk}\hlstd{(dat)}
\end{alltt}
\end{kframe}
\end{knitrout}

Raw nutrient data obtained from the \ac{CDMO} will usually include replicate samples that were taken within a few minutes of each other.  The \texttt{rem\_reps.swmpr} function combines nutrient data that occur on the same day to preserve an approximate monthly time step.  The \texttt{datetimestamp} column will always be averaged for replicates, but the actual observations will be combined based on the user-supplied function which defauls to the mean.  Other suggested functions include the \texttt{median}, \texttt{min}, or \texttt{max}.  The entire function call including treatment of NA values should be passed to the \texttt{FUN} argument (see the examples).  The function is meant to be used after \texttt{qaqc} processing, although it works with a warning if QAQC columns are present.

\begin{knitrout}
\definecolor{shadecolor}{rgb}{0.969, 0.969, 0.969}\color{fgcolor}\begin{kframe}
\begin{alltt}
\hlcom{# get nutrient data}
\hlkwd{data}\hlstd{(apacpnut)}
\hlstd{swmp1} \hlkwb{<-} \hlstd{apacpnut}
\hlstd{swmp1} \hlkwb{<-} \hlkwd{qaqc}\hlstd{(swmp1)}

\hlcom{# remove replicate nutrient data}
\hlkwd{rem_reps}\hlstd{(swmp1)}

\hlcom{# use different function to aggregate replicates}
\hlstd{func} \hlkwb{<-} \hlkwa{function}\hlstd{(}\hlkwc{x}\hlstd{)} \hlkwd{max}\hlstd{(x,} \hlkwc{na.rm} \hlstd{= T)}
\hlkwd{rem_reps}\hlstd{(swmp1,} \hlkwc{FUN} \hlstd{= func)}
\end{alltt}
\end{kframe}
\end{knitrout}

A subset method added to the existing generic \texttt{subset} function in R is available for \texttt{swmpr} objects.  This function is used to subset the data by date and/or a selected parameter.  The date can be a single value or as two dates to select records within the range. The former case requires a binary operator input as a character string passed to the argument, such as \texttt{>} or \texttt{<}.  The subset argument for the date(s) must also be a character string of the format YYYY-mm-dd HH:MM for each element (i.e., \%Y-\%m\%-\%d \%H:\%M in POSIX standards).  Be aware that an error may be returned using this function if the subset argument is in the correct format but the calendar date does not exist, e.g. \texttt{2012-11-31 12:00}.  Finally, the function can be used to remove rows and columns that do not contain data. 

\begin{knitrout}
\definecolor{shadecolor}{rgb}{0.969, 0.969, 0.969}\color{fgcolor}\begin{kframe}
\begin{alltt}
\hlcom{# select two parameters from dat}
\hlkwd{subset}\hlstd{(dat,} \hlkwc{select} \hlstd{=} \hlkwd{c}\hlstd{(}\hlstr{'rh'}\hlstd{,} \hlstr{'bp'}\hlstd{))}

\hlcom{# subset records greater than or equal to a date}
\hlkwd{subset}\hlstd{(dat,} \hlkwc{subset} \hlstd{=} \hlstr{'2013-01-01 0:00'}\hlstd{,} \hlkwc{operator} \hlstd{=} \hlstr{'>='}\hlstd{)}

\hlcom{# subset records within a date range}
\hlkwd{subset}\hlstd{(dat,} \hlkwc{subset} \hlstd{=} \hlkwd{c}\hlstd{(}\hlstr{'2012-07-01 6:00'}\hlstd{,} \hlstr{'2012-08-01 18:15'}\hlstd{))}

\hlcom{# subset records within a date range, select two parameters}
\hlkwd{subset}\hlstd{(dat,} \hlkwc{subset} \hlstd{=} \hlkwd{c}\hlstd{(}\hlstr{'2012-07-01 6:00'}\hlstd{,} \hlstr{'2012-08-01 18:15'}\hlstd{),}
  \hlkwc{select} \hlstd{=} \hlkwd{c}\hlstd{(}\hlstr{'atemp'}\hlstd{,} \hlstr{'totsorad'}\hlstd{))}

\hlcom{# remove rows/columns that do not contain data}
\hlkwd{subset}\hlstd{(dat,} \hlkwc{rem_rows} \hlstd{= T,} \hlkwc{rem_cols} \hlstd{= T)}
\end{alltt}
\end{kframe}
\end{knitrout}

The \texttt{setstep} function formats a \texttt{swmpr} object to a continuous time series at a given time step.  This function is not necessary for most stations but can be useful for combining data or converting an existing time series to a set interval.  The first argument of the function, \texttt{timestep}, specifies the desired time step in minutes starting from the nearest hour of the first observation.  The second argument, \texttt{differ}, specifies the allowable tolerance in minutes for matching existing observations to user-defined time steps in cases where the two are dissimilar.  Values for \texttt{differ} that are greater than one half the value of \texttt{timestep} are not allowed to prevent duplication of existing data.  Likewise, the default value for \texttt{differ} is one half the time step.  Rows that do not match any existing data within the limits of the \texttt{differ} argument are not discarded.  Output from the \texttt{setstep} function can be used with \texttt{subset} and to create a time series at a set interval with empty data removed.

\begin{knitrout}
\definecolor{shadecolor}{rgb}{0.969, 0.969, 0.969}\color{fgcolor}\begin{kframe}
\begin{alltt}
\hlcom{# convert time series to two hour invervals}
\hlcom{# tolerance of +/- 30 minutes for matching existing data}
\hlkwd{setstep}\hlstd{(dat,} \hlkwc{timestep} \hlstd{=} \hlnum{120}\hlstd{,} \hlkwc{differ} \hlstd{=} \hlnum{30}\hlstd{)}

\hlcom{# convert a nutrient time series to a continuous time series}
\hlcom{# then remove empty rows and columns}
\hlkwd{data}\hlstd{(apacpnut)}
\hlstd{dat_nut} \hlkwb{<-} \hlstd{apacpnut}
\hlstd{dat_nut} \hlkwb{<-} \hlkwd{setstep}\hlstd{(dat_nut,} \hlkwc{timestep} \hlstd{=} \hlnum{60}\hlstd{)}
\hlkwd{subset}\hlstd{(dat_nut,} \hlkwc{rem_rows} \hlstd{= T,} \hlkwc{rem_cols} \hlstd{= T)}
\end{alltt}
\end{kframe}
\end{knitrout}

The \texttt{comb} function is used to combine multiple \texttt{swmpr} objects into a single object with a continuous time series at a given step.  The \texttt{timestep} function is used internally such that \texttt{timestep} and \texttt{differ} are accepted arguments for \texttt{comb}.  The function requires one or more \texttt{swmpr} objects as input as separate, undefined arguments.  The remaining arguments must be called explicitly since an arbitrary number of objects can be used as input.  In general, the function combines data by creating a master time series that is used to iteratively merge all \texttt{swmpr} objects.  The time series for merging depends on the value passed to the \texttt{method} argument.  Passing \texttt{union} to \texttt{method} will create a time series that is continuous starting from the earliest date and the latest date for all input objects.  Passing \texttt{intersect} to \texttt{method} will create a time series that is continuous from the set of dates that are shared between all input objects.  Finally, a seven or eight character station name passed to \texttt{method} will merge all input objects based on a continuous time series for the given station.  The specified station must be present in the input data.  Currently, combining data types from different stations is not possible, excluding weather data which are typically at a single, dedicated station.  

\begin{knitrout}
\definecolor{shadecolor}{rgb}{0.969, 0.969, 0.969}\color{fgcolor}\begin{kframe}
\begin{alltt}
\hlcom{# get nuts, wq, and met data as separate objects for the same station}
\hlcom{# note that most sites usually have one weather station}
\hlkwd{data}\hlstd{(apacpnut)}
\hlkwd{data}\hlstd{(apacpwq)}
\hlkwd{data}\hlstd{(apaebmet)}
\hlstd{swmp1} \hlkwb{<-} \hlstd{apacpnut}
\hlstd{swmp2} \hlkwb{<-} \hlstd{apacpwq}
\hlstd{swmp3} \hlkwb{<-} \hlstd{apaebmet}

\hlcom{# combine nuts and wq data by union}
\hlkwd{comb}\hlstd{(swmp1, swmp2,} \hlkwc{method} \hlstd{=} \hlstr{'union'}\hlstd{)}

\hlcom{# combine nuts and wq data by intersect}
\hlkwd{comb}\hlstd{(swmp1, swmp3,} \hlkwc{method} \hlstd{=} \hlstr{'intersect'}\hlstd{)}

\hlcom{# combine nuts, wq, and met data by nuts time series, two hour time step}
\hlkwd{comb}\hlstd{(swmp1, swmp2, swmp3,} \hlkwc{timestep} \hlstd{=} \hlnum{120}\hlstd{,} \hlkwc{method} \hlstd{=} \hlstr{'apacpnut'}\hlstd{)}
\end{alltt}
\end{kframe}
\end{knitrout}

\subsection*{Data analysis}

The analysis functions range from general purpose tools for time series analysis to more specific functions for working with continuous monitoring data in estuaries (\Cref{tab:analyze}).  The latter category includes a limited number of functions that were developed by myself or others.  The general purpose tools are \texttt{swmpr} methods that were developed for existing generic functions in the R base installation or relevant packages.  These functions include \texttt{swmpr} methods for \texttt{aggregate}, \texttt{filter}, and \texttt{approx} to deal with missing or noisy data and more general functions for exploratory data analaysis, such as \texttt{plot}, \texttt{summary}, and \texttt{hist} methods.  Decomposition functions, \texttt{decomp} and \texttt{decomp\_cj}, are provided as relatively simple approaches for decomposing time series into additive or multiplicative components. The analysis functions may or may not return a \texttt{swmpr} object depending on whether further processing with \texttt{swmpr} methods is possible from the output.    

The \texttt{aggregate} function aggregates parameter data for a \texttt{swmpr} object by set periods of observation.  This function is most useful for aggregating noisy data to evaluate trends on longer time scales, or to simply reduce the size of a dataset.  Data can be aggregated by years, quarters, months, weeks, days, or hours for a user-defined function, which defaults to the mean.  A \texttt{swmpr} object is returned for the aggregated data, although the \texttt{datetimestamp} vector will be converted to a date object if the aggregation period is a day or longer.  Days are assigned to the date vector if the aggregation period is a week or longer based on the \texttt{round} method for IDate objects \href{http://cran.r-project.org/web/packages/data.table/index.html}{data.table} package.  This approach was used to facilitate plotting using predefined methods for Date and POSIX objects.  Additionally, the method of treating NA values for the aggregation function should be noted since this may greatly affect the quantity of data that are returned (see the example below).  Finally, the default argument for \texttt{na.action} is set to \texttt{na.pass} for \texttt{swmpr} objects to preserve the time series of the input data.

% table for analysis functions
%latex.default(to_tab[, "Description", drop = F], file = "", caption = "Analysis functions available from the SWMPr package.  Full documentation for each function is in the help file (e.g., execute \\texttt{?aggregate.swmpr} at the command line).",     rowlabel = "Function", colheads = "Description", rowname = to_tab$Functions,     caption.loc = "top", col.just = c("p{3.5in}"), label = "tab:analyze",     table.env = FALSE)%
\begin{table}[!tbp]
\caption{Analysis functions available from the SWMPr package.  Full documentation for each function is in the help file (e.g., execute \texttt{?aggregate.swmpr} at the command line).\label{tab:analyze}} 
\begin{center}
\begin{tabular}{lp{3.5in}}
\hline\hline
\multicolumn{1}{l}{Function}&\multicolumn{1}{c}{Description}\tabularnewline
\hline
\texttt{aggregate.swmpr}&Aggregate \texttt{swmpr} objects for different time periods - years, quarters, months,  weeks, days, or hours.  Aggregation function is user-supplied but defaults to mean.\tabularnewline
\texttt{aggregate\_metab.swmpr}&Aggregate metabolism data from a \texttt{swmpr} object.  This is primarly used within \texttt{plot\_metab} but may be useful for simple summaries of raw daily data.\tabularnewline
\texttt{ecometab.swmpr}&Estimate ecosystem metabolism for a combined water quality and weatehr dataset using the open-water method.\tabularnewline
\texttt{decomp.swmpr}&Decompose a \texttt{swmpr} time series into trend, seasonal, and residual components.  This is a simple wrapper to \texttt{decompose}.  Decomposition of monthly or daily trends is possible.\tabularnewline
\texttt{decomp\_cj.swmpr}&Decompose a \texttt{swmpr} time series into grandmean, annual, seasonal, and events components.  This is a simple wrapper to \texttt{decompTs} in the wq package.  Only monthly decomposition is possible.\tabularnewline
\texttt{hist.swmpr}&Plot a histogram for a \texttt{swmpr} object.\tabularnewline
\texttt{lines.swmpr}&Add lines to an existing \texttt{swmpr} plot.\tabularnewline
\texttt{na.approx.swmpr}&Linearly interpolate missing data (\texttt{NA} values) in a \texttt{swmpr} object. The maximum gap size that is interpolated is defined as a maximum number of records with missing data.\tabularnewline
\texttt{plot.swmpr}&Plot a univariate  time series for a \texttt{swmpr} object.  The parameter name must be specified.\tabularnewline
\texttt{plot\_metab.swmpr}&Plot ecosystem metabolism estimates after running \texttt{ecometab} on a \texttt{swmpr} object.\tabularnewline
\texttt{plot\_summary.swmpr}&Create summary plots of seasonal/annual trends and anomalies for a water quality or weather parameter.\tabularnewline
\texttt{smoother.swmpr}&Smooth \texttt{swmpr} objects with a moving window average.  Window size and sides can be specified, passed to \texttt{filter}.\tabularnewline
\hline
\end{tabular}\end{center}

\end{table}


\begin{knitrout}
\definecolor{shadecolor}{rgb}{0.969, 0.969, 0.969}\color{fgcolor}\begin{kframe}
\begin{alltt}
\hlcom{# combine, qaqc, remove empty columns}
\hlstd{dat} \hlkwb{<-} \hlkwd{comb}\hlstd{(swmp1, swmp2,} \hlkwc{method} \hlstd{=} \hlstr{'union'}\hlstd{)}
\hlstd{dat} \hlkwb{<-} \hlkwd{qaqc}\hlstd{(dat)}
\hlstd{swmpr_in} \hlkwb{<-} \hlkwd{subset}\hlstd{(dat,} \hlkwc{rem_cols} \hlstd{= T)}

\hlcom{# get mean DO by quarters}
\hlkwd{aggregate}\hlstd{(swmpr_in,} \hlstr{'quarters'}\hlstd{,} \hlkwc{params} \hlstd{=} \hlkwd{c}\hlstd{(}\hlstr{'do_mgl'}\hlstd{))}

\hlcom{# get mean DO by quarters, remove NA when calculating means}
\hlstd{fun_in} \hlkwb{<-} \hlkwa{function}\hlstd{(}\hlkwc{x}\hlstd{)} \hlkwd{mean}\hlstd{(x,} \hlkwc{na.rm} \hlstd{= T)}
\hlkwd{aggregate}\hlstd{(swmpr_in,} \hlkwc{FUN} \hlstd{= fun_in,} \hlstr{'quarters'}\hlstd{,} \hlkwc{params} \hlstd{=} \hlkwd{c}\hlstd{(}\hlstr{'do_mgl'}\hlstd{))}
\end{alltt}
\end{kframe}
\end{knitrout}

Time series can be smoothed to better characterize a signal independent of noise (\Cref{fig:smooth_ex}).  Although there are many approaches to smoothing, a moving window average is intuitive and commonly used.  The \texttt{smoother} function can be used to smooth parameters in a \texttt{swmpr} object using a specified window size.  This method is a simple wrapper to \texttt{filter}.  The \texttt{window} argument specifies the number of observations included in the moving average.  The \texttt{sides} argument specifies how the average is calculated for each observation (see the documentation for \texttt{filter}).  A value of 1 will filter observations within the window that are previous to the current observation, whereas a value of 2 will filter all observations within the window centered at zero lag from the current observation. As before, the \texttt{params} argument specifies which parameters to smooth.  See \Cref{fig:smooth_ex} for the output from the code.

\begin{knitrout}
\definecolor{shadecolor}{rgb}{0.969, 0.969, 0.969}\color{fgcolor}\begin{kframe}
\begin{alltt}
\hlcom{# import data}
\hlkwd{data}\hlstd{(apadbwq)}
\hlstd{swmp1} \hlkwb{<-} \hlstd{apadbwq}

\hlcom{# qaqc and subset imported data}
\hlstd{dat} \hlkwb{<-} \hlkwd{qaqc}\hlstd{(swmp1)}
\hlstd{dat} \hlkwb{<-} \hlkwd{subset}\hlstd{(dat,} \hlkwc{subset} \hlstd{=} \hlkwd{c}\hlstd{(}\hlstr{'2012-07-09 00:00'}\hlstd{,} \hlstr{'2012-07-24 00:00'}\hlstd{))}

\hlcom{# filter}
\hlstd{test} \hlkwb{<-} \hlkwd{smoother}\hlstd{(dat,} \hlkwc{window} \hlstd{=} \hlnum{50}\hlstd{,} \hlkwc{params} \hlstd{=} \hlstr{'do_mgl'}\hlstd{)}

\hlcom{# plot to see the difference}
\hlkwd{plot}\hlstd{(do_mgl} \hlopt{~} \hlstd{datetimestamp,} \hlkwc{data} \hlstd{= dat,} \hlkwc{type} \hlstd{=} \hlstr{'l'}\hlstd{)}
\hlkwd{lines}\hlstd{(test,} \hlkwc{select} \hlstd{=} \hlstr{'do_mgl'}\hlstd{,} \hlkwc{col} \hlstd{=} \hlstr{'red'}\hlstd{,} \hlkwc{lwd} \hlstd{=} \hlnum{2}\hlstd{)}
\end{alltt}
\end{kframe}\begin{figure}[!ht]


{\centering \includegraphics[width=\maxwidth]{figure/smooth_ex} 

}

\caption[Raw and smoothed dissolved oxygen data for a two-week period after using the \texttt{smoother} function]{Raw and smoothed dissolved oxygen data for a two-week period after using the \texttt{smoother} function.\label{fig:smooth_ex}}
\end{figure}


\end{knitrout}

A common issue with any statistical analysis is the treatment of missing values.  Missing data can be excluded from the analysis, included but treated as true zeroes, or interpolated based on similar values.  In either case, an analyst should have a strong rationale for the chosen method.  A common approach used to handle missing data in time series analysis is linear interpolation.  A simple curve fitting method is used to create a continuous set of records between observations separated by missing data.  A challenge with linear interpolation is an appropriate gap size for fitting missing observations.  The ability of the interpolated data to approximate actual trends is a function of the gap size.  Interpolation between larger gaps are less likely to resemble patterns of an actual parameter, whereas interpolation between smaller gaps are more likely to resemble actual patterns.  An appropriate gap size limit depends on the unique characteristics of specific datasets or parameters.  The \texttt{na.approx} function can be used to interpolate gaps in a \texttt{swmpr} object.  A required argument for the function is \texttt{maxgap} which defines the maximum gap size for interpolation.  See \Cref{fig:interp_ex} for the output from the following code.

\begin{knitrout}
\definecolor{shadecolor}{rgb}{0.969, 0.969, 0.969}\color{fgcolor}\begin{kframe}
\begin{alltt}
\hlcom{# get data}
\hlkwd{data}\hlstd{(apadbwq)}
\hlstd{swmp1} \hlkwb{<-} \hlstd{apadbwq}

\hlcom{# qaqc and subset imported data}
\hlstd{dat} \hlkwb{<-} \hlkwd{qaqc}\hlstd{(swmp1)}
\hlstd{dat} \hlkwb{<-} \hlkwd{subset}\hlstd{(dat,} \hlkwc{subset} \hlstd{=} \hlkwd{c}\hlstd{(}\hlstr{'2013-01-22 00:00'}\hlstd{,} \hlstr{'2013-01-26 00:00'}\hlstd{))}

\hlcom{# interpolate, maxgap of 10 records}
\hlstd{test} \hlkwb{<-} \hlkwd{na.approx}\hlstd{(dat,} \hlkwc{params} \hlstd{=} \hlstr{'do_mgl'}\hlstd{,} \hlkwc{maxgap} \hlstd{=} \hlnum{10}\hlstd{)}

\hlcom{# interpolate maxgap of 30 records}
\hlstd{test2} \hlkwb{<-} \hlkwd{na.approx}\hlstd{(dat,} \hlkwc{params} \hlstd{=} \hlstr{'do_mgl'}\hlstd{,} \hlkwc{maxgap} \hlstd{=} \hlnum{30}\hlstd{)}

\hlcom{# plot for comparison}
\hlkwd{par}\hlstd{(}\hlkwc{mfrow} \hlstd{=} \hlkwd{c}\hlstd{(}\hlnum{3}\hlstd{,} \hlnum{1}\hlstd{))}
\hlkwd{plot}\hlstd{(do_mgl} \hlopt{~} \hlstd{datetimestamp, dat,} \hlkwc{main} \hlstd{=} \hlstr{'Raw'}\hlstd{,} \hlkwc{type} \hlstd{=} \hlstr{'l'}\hlstd{)}
\hlkwd{plot}\hlstd{(do_mgl} \hlopt{~} \hlstd{datetimestamp, test,} \hlkwc{col} \hlstd{=} \hlstr{'red'}\hlstd{,}
  \hlkwc{main} \hlstd{=} \hlstr{'Interpolation - maximum gap of 10 records'}\hlstd{,} \hlkwc{type} \hlstd{=} \hlstr{'l'}\hlstd{)}
\hlkwd{lines}\hlstd{(dat,} \hlkwc{select} \hlstd{=} \hlstr{'do_mgl'}\hlstd{)}
\hlkwd{plot}\hlstd{(do_mgl} \hlopt{~} \hlstd{datetimestamp, test2,} \hlkwc{col} \hlstd{=} \hlstr{'red'}\hlstd{,}
  \hlkwc{main} \hlstd{=} \hlstr{'Interpolation - maximum gap of 30 records'}\hlstd{,} \hlkwc{type} \hlstd{=} \hlstr{'l'}\hlstd{)}
\hlkwd{lines}\hlstd{(dat,} \hlkwc{select} \hlstd{=} \hlstr{'do_mgl'}\hlstd{)}
\end{alltt}
\end{kframe}\begin{figure}[!ht]


{\centering \includegraphics[width=\maxwidth]{figure/interp_ex} 

}

\caption[Examples illustrating use of the \texttt{na.approx} function to fill gaps of different sizes in a dissolved oxygen time series for a four day period.]{Examples illustrating use of the \texttt{na.approx} function to fill gaps of different sizes in a dissolved oxygen time series for a four day period.\label{fig:interp_ex}}
\end{figure}


\end{knitrout}

The \texttt{decomp} function is a simple wrapper to \texttt{decompose} that separates a time series into additive or multiplicative components describing a trend, cyclical variation (e.g., daily or seasonal), and the remainder.  The additive decomposition assumes that the cyclical component of the time series is stationary (i.e., the variance is constant), whereas a multiplicative decomposition accounts for non-stationarity.  By default, a moving average with a symmetric window is used to filter the seasonal component.  Alternatively, a vector of filter coefficients in reverse time order can be supplied (see the help documentation for \texttt{decompose}).  

The \texttt{decompose} function requires a \texttt{ts} object with a specified frequency as input.  The \texttt{decomp} function converts the input \texttt{swmpr} vector to a \texttt{ts} object prior to \texttt{decompose}.  This requires an explicit input defining the frequency of the parameter in the time series.  For example, the frequency of a parameter with diurnal periodicity would be 96 if the time step is 15 minutes (4 * 24).  The frequency of a parameter with seasonal periodicity would be 35040 (4 * 24 * 365).  For simplicity, character strings of \texttt{`daily'} or \texttt{`seasonal'} can be supplied in place of numeric values.  A starting value of the time series must be supplied in the latter case.  Use of the \texttt{setstep} function is also required to standardize the time step prior to decomposition.  Note that the \texttt{decompose} function is a relatively simple approach and alternative methods should be investigated if a more sophisticated decomposition is desired.  \Cref{fig:decomp_ex1} is an example of the \texttt{decomp} function.

\begin{knitrout}
\definecolor{shadecolor}{rgb}{0.969, 0.969, 0.969}\color{fgcolor}\begin{kframe}
\begin{alltt}
\hlcom{# get data}
\hlkwd{data}\hlstd{(apadbwq)}
\hlstd{swmp1} \hlkwb{<-} \hlstd{apadbwq}

\hlcom{# subset for daily decomposition}
\hlstd{dat} \hlkwb{<-} \hlkwd{subset}\hlstd{(swmp1,} \hlkwc{subset} \hlstd{=} \hlkwd{c}\hlstd{(}\hlstr{'2013-07-01 00:00'}\hlstd{,} \hlstr{'2013-07-31 00:00'}\hlstd{))}

\hlcom{# decomposition and plot}
\hlstd{test} \hlkwb{<-} \hlkwd{decomp}\hlstd{(dat,} \hlkwc{param} \hlstd{=} \hlstr{'do_mgl'}\hlstd{,} \hlkwc{frequency} \hlstd{=} \hlstr{'daily'}\hlstd{)}
\hlkwd{plot}\hlstd{(test)}
\end{alltt}
\end{kframe}\begin{figure}[!ht]


{\centering \includegraphics[width=\maxwidth]{figure/decomp_ex1} 

}

\caption[An additive decomposition of dissolved oxygen into a trend, seasonal, and random component using the \texttt{decomp} function]{An additive decomposition of dissolved oxygen into a trend, seasonal, and random component using the \texttt{decomp} function.\label{fig:decomp_ex1}}
\end{figure}


\end{knitrout}

An alternative approach to time series decomposition is provided by the \texttt{decomp\_cj} function, which is a simple wrapper to the \texttt{decompTs} function in the wq package.  Theory describing this method is provided by Cloern and Jassby \cite{Cloern10}.  The function is similar to \texttt{decomp.swmpr} with a few key differences.  The \texttt{decomp.swmpr} function decomposes the time series into a trend, seasonal, and random component, whereas the current function decomposes into the grandmean, annual, seasonal, and events components.  For both functions, the random or events components, respectively, can be considered anomalies that do not follow the trends in the remaining categories.  The \texttt{decomp\_cj} function provides only a monthly decomposition, which is appropriate for characterizing relatively long-term trends.  This approach works best for nutrient data that are typically obtained on a monthly cycle.  The function will also work with continuous water quality or weather data but note that the data must first aggregated on the monthly scale before decomposition.  Additional arguments passed to \texttt{decompTs} can be used with \texttt{decomp\_cj}, such as \texttt{startyr}, \texttt{endyr}, and \texttt{type}.  Values passed to \texttt{type} are \texttt{mult} (default) or \texttt{add}, referring to multiplicative or additive decomposition.  \Cref{fig:decomp_ex2} shows the results from the \texttt{decomp\_cj} function applied to a multi-year chlorophyll time series.

\begin{knitrout}
\definecolor{shadecolor}{rgb}{0.969, 0.969, 0.969}\color{fgcolor}\begin{kframe}
\begin{alltt}
\hlcom{# get data}
\hlkwd{data}\hlstd{(apacpnut)}
\hlstd{dat} \hlkwb{<-} \hlstd{apacpnut}
\hlstd{dat} \hlkwb{<-} \hlkwd{qaqc}\hlstd{(dat,} \hlkwc{qaqc_keep} \hlstd{=} \hlkwa{NULL}\hlstd{)}

\hlcom{# decomposition of chl, ggplot}
\hlkwd{decomp_cj}\hlstd{(dat,} \hlkwc{param} \hlstd{=} \hlstr{'chla_n'}\hlstd{)}
\end{alltt}
\end{kframe}\begin{figure}[!ht]


{\centering \includegraphics[width=\maxwidth]{figure/decomp_ex2} 

}

\caption[Additive decomposition of a multi-year chlorophyll time series into the grandmean, annual, seasonal, and events components using the \texttt{decomp\_cj} function]{Additive decomposition of a multi-year chlorophyll time series into the grandmean, annual, seasonal, and events components using the \texttt{decomp\_cj} function.\label{fig:decomp_ex2}}
\end{figure}


\end{knitrout}

Several graphics showing seasonal and annual trends for a given SWMP parameter can be obtained using the \texttt{plot\_summary} function.  The plots include monthly distributions, monthly anomalies, and annual anomalies in multiple formats.  Anomalies are defined as the difference between the monthly or annual average from the grand mean for the parameter.  Monthly anomalies are in relation to the grand mean for the same month across all years.  All data are aggregated for quicker plotting.  Nutrient data are based on monthly averages, whereas weather and water quality data are based on daily averages.  Cumulative precipitation data are based on the daily maximum. The function returns a graphics object (Grob) of multiple ggplot objects.  An interactive Shiny application \cite{shiny15} that uses this function is available (see the \nameref{supp_info}).

\begin{knitrout}
\definecolor{shadecolor}{rgb}{0.969, 0.969, 0.969}\color{fgcolor}\begin{kframe}
\begin{alltt}
\hlcom{## import data}
\hlkwd{data}\hlstd{(apacpnut)}
\hlstd{dat} \hlkwb{<-} \hlkwd{qaqc}\hlstd{(apacpnut)}

\hlcom{## plot}
\hlkwd{plot_summary}\hlstd{(dat,} \hlkwc{param} \hlstd{=} \hlstr{'chla_n'}\hlstd{,} \hlkwc{years} \hlstd{=} \hlkwd{c}\hlstd{(}\hlnum{2007}\hlstd{,} \hlnum{2013}\hlstd{))}
\end{alltt}
\end{kframe}\begin{figure}[!ht]


{\centering \includegraphics[width=\maxwidth]{figure/summary_ex} 

}

\caption[Summaries of a multi-year chlorophyll time series using the \texttt{plot\_summary} function]{Summaries of a multi-year chlorophyll time series using the \texttt{plot\_summary} function.  Summaries include monthly distributions (means on top left, quantiles on bottom left), monthly histograms (center), monthly means by year (top right), deviation from monthly means (middle right), and annual trends as deviations from the grand mean (bottom right)\label{fig:summary_ex}}
\end{figure}


\end{knitrout}

Estimates of ecosystem metabolism provide a measure of overall system productivity to evaluate whether an ecosystem is a net source or sink of organic material.  The open-water method \cite{Odum56} is a common approach to quantify metabolism using a mass balance equation that describes the change in dissolved oxygen over time from the balance between photosynthetic and respiration processes, corrected using an empirically constrained air-sea gas diffusion model (see Ro and Hunt \cite{Ro06}, Thebault et al. \cite{Thebault08}). The diffusion-corrected \ac{DO} flux estimates are averaged separately over each day and night of the time series. The nighttime average \ac{DO} flux is used to estimate respiration rates, while the daytime \ac{DO} flux is used to estimate net primary production. To generate daily integrated rates, respiration rates are assumed constant such that hourly night time \ac{DO} flux rates are multiplied by 24. Similarly, the daytime \ac{DO} flux rates are multiplied by the number of daylight hours, which varies with location and time of year, to yield net daytime primary production. Respiration rates are subtracted from daily net production estimates to yield gross production rates. The metabolic day is considered the 24 hour period between sunsets on two adjacent calendar days  

The \texttt{ecometab} function is used to implement the open-water method with a combined water quality and weather dataset \cite{Caffrey14}.  Several assumptions must be met for a valid interpretation of the results.  In general, the dissolved oxygen time series is assumed to represent the same water mass over time.  Tidal advection may have a significant influence on the time series, which can contribute to a significant amount of noise in metabolic estimates.  The extent to which tidal advection influences the dissolved oxygen signal depends on various site-level characteristics and an intimate knowledge of the site may be required.  Areal rates for gross production and total respiration are based on volumetric rates normalized to the depth of the water column at the sampling location, which is assumed to be well-mixed, such that the water quality sensor is reflecting the integrated processes in the entire water column (including the benthos).  Water column depth is calculated as the mean value of the depth variable across the time series in the \texttt{swmpr} object. Depth values are floored at one meter for very shallow stations and 0.5 meters is also added to reflect the practice of placing sensors slightly off of the bottom.  Additionally, the air-sea gas exchange model is calibrated with wind data either collected at, or adjusted to, wind speed at 10 m above the surface.  The metadata should be consulted for exact height. Other assumptions may apply and relevant resources should be consulted to ensure appropriate application of the open-water method (see \cite{Kemp12,Needoba12}).

The following is an example that shows use of the function from a combined water quality and weather data set.  The results can be plotted using \texttt{plot\_metab} (\Cref{fig:metab_ex}).

\begin{knitrout}
\definecolor{shadecolor}{rgb}{0.969, 0.969, 0.969}\color{fgcolor}\begin{kframe}
\begin{alltt}
\hlcom{## import water quality and weather data}
\hlkwd{data}\hlstd{(apadbwq)}
\hlkwd{data}\hlstd{(apaebmet)}

\hlcom{## qaqc, combine}
\hlstd{wq} \hlkwb{<-} \hlkwd{qaqc}\hlstd{(apadbwq)}
\hlstd{met} \hlkwb{<-} \hlkwd{qaqc}\hlstd{(apaebmet)}
\hlstd{dat} \hlkwb{<-} \hlkwd{comb}\hlstd{(wq, met)}

\hlcom{## estimate metabolism}
\hlstd{res} \hlkwb{<-} \hlkwd{ecometab}\hlstd{(dat,} \hlkwc{trace} \hlstd{=} \hlnum{FALSE}\hlstd{)}
\hlkwd{plot_metab}\hlstd{(res)}
\end{alltt}
\end{kframe}\begin{figure}[!ht]


{\centering \includegraphics[width=\maxwidth]{figure/metab_ex} 

}

\caption[Monthly aggregations of ecosystem metabolism estimates (net ecosystem metabolism, gross production, and total respiration) for combined water quality and weather data at Apalachicola Bay, Florida]{Monthly aggregations of ecosystem metabolism estimates (net ecosystem metabolism, gross production, and total respiration) for combined water quality and weather data at Apalachicola Bay, Florida.\label{fig:metab_ex}}
\end{figure}


\end{knitrout}

\subsection*{Miscellaneous functions}

Several additional functions are provided that do not fit the above categories (\Cref{tab:misc}).  These functions are generally used within the main functions but may be useful for more customized evaluation of \ac{SWMP} data.  

% table for misc functions
%latex.default(to_tab[, "Description", drop = F], file = "", caption = "Miscellaneous functions available from the SWMPr package.  Most are used within the main functions above but may be useful for more customized evaluationg of \\ac{SWMP} data.  Full documentation for each function is in the help file (e.g., execute \\texttt{?calckl} at the command line).",     rowlabel = "Function", colheads = "Description", rowname = to_tab$Functions,     caption.loc = "top", col.just = c("p{3.5in}"), label = "tab:misc",     table.env = FALSE)%
\begin{table}[!tbp]
\caption{Miscellaneous functions available from the SWMPr package.  Most are used within the main functions above but may be useful for more customized evaluationg of \ac{SWMP} data.  Full documentation for each function is in the help file (e.g., execute \texttt{?calckl} at the command line).\label{tab:misc}} 
\begin{center}
\begin{tabular}{lp{3.5in}}
\hline\hline
\multicolumn{1}{l}{Function}&\multicolumn{1}{c}{Description}\tabularnewline
\hline
\texttt{calckl}&Estimate the reaeration coefficient for air-sea gas exchange.  This is only used within the \texttt{ecometab} function.\tabularnewline
\texttt{map\_reserve}&Create a map of all stations in a reserve using the ggmap package.\tabularnewline
\texttt{metab\_day}&Identify the metabolic day for each approximate 24 period in an hourly time series.  This is only used within the \texttt{ecometab} function.\tabularnewline
\texttt{param\_names}&Returns column names as a list for the parameter type(s) (nutrients, weather, or water quality).  Includes \ac{QAQC} columns with \texttt{f\_} prefix. Used internally in other functions.\tabularnewline
\texttt{parser}&Parses html returned from \ac{CDMO} web services, used internally in retrieval functions.\tabularnewline
\texttt{swmpr}&Creates object of \texttt{swmpr} class, used internally in retrieval functions.\tabularnewline
\texttt{time\_vec}&Converts time vectors to \texttt{POSIXct} objects with correct time zone for a site/station, used internally in retrieval functions.\tabularnewline
\hline
\end{tabular}\end{center}

\end{table}


For brevity, only the \texttt{reserve\_map} function is discussed.  This function can be used to create a map with all stations at a reserve by passing arguments to functions in the ggmap package \cite{ggmap13}. The current function is limited to Google maps of four types that can be set with the \texttt{map\_type} argument: terrain (default), satellite, roadmap, or hybrid.  The \texttt{zoom} argument may have to be chosen through trial and error depending on the spatial extent of the reserve.  See the help documentation for the \texttt{ggmap} function for more info on \texttt{zoom}.

\begin{knitrout}
\definecolor{shadecolor}{rgb}{0.969, 0.969, 0.969}\color{fgcolor}\begin{kframe}
\begin{alltt}
\hlcom{# plot the stations at Jacques Cousteau reserve}
\hlkwd{map_reserve}\hlstd{(}\hlstr{'jac'}\hlstd{)}
\end{alltt}
\end{kframe}\begin{figure}[!ht]


{\centering \includegraphics[width=\maxwidth]{figure/map_ex} 

}

\caption[Locations of all sites at the Jacques Cousteau reserve using the \texttt{map\_reserve} function]{Locations of all sites at the Jacques Cousteau reserve using the \texttt{map\_reserve} function.\label{fig:map_ex}}
\end{figure}


\end{knitrout}

\section*{Applications using the SWMPr package}

The ability to evaluate environmental characteristics between estuaries within the \ac{NERRS} program has been greatly improved using functions in the SWMPr package.  This section describes three examples using the SWMPr package to illustrate the improved ability to synthesize and evaluate multi-year time series of estuarine data.  

First, the open-water method for estimating metabolism was applied to nearly all co-located water quality and weather sites at the \ac{NERRS} reserve for all years of available data.  The results are provided primarily to illustrate ease of use of the functions and secondarily to provide an update to results described in Caffrey \cite{Caffrey03} and Caffrey \cite{Caffrey04} using the most recent \ac{SWMP} data.  Additionally, previous methods to evaluate metabolism at \ac{NERRS} reserves did not incorporate \textit{in situ} weather data to estimate the air-sea gas exchange model.  Caffrey \cite{Caffrey03} and Caffrey \cite{Caffrey04} applied the open-water method to estimate ecosystem metabolism using five years of water quality observations at two sites at each of the \ac{NERRS} reserves.  Additionally, the air-sea gas exchange model assumed a constant value for the reaeratio coefficientt.  This coefficient provides an estimate of the rate of air-sea gas exchange that varies as a function of wind speed, temperature, barometric pressure, salinity, and depth of the water column.  The inclusion of weather data in the calculation allows for a more precise estimate of air-sea gas exchange and consequently more reliable estimates of ecosystem metabolism (see Caffrey et al. \cite{Caffrey14} for details).

Water quality and weather observations from January 1995 to December 2014 for all \ac{NERRS} sites were obtained through a bulk data request using the \href{http://cdmo.baruch.sc.edu/aqs/zips.cfm}{zip downloads} feature of \ac{CDMO}.  All csv files for each station were imported into R using the \texttt{import\_local} function, processed using the \texttt{setstep} and \texttt{qaqc} functions, then saved on a local hard drive as binary RData files (see \href{https://github.com/fawda123/swmp_rats/blob/master/R/dat_proc.R}{here} for the R script). This resulted in a single \texttt{swmpr} object for each parameter at each site.  All files were then uploaded to a remote server.  An additional \href{https://gist.github.com/fawda123/4fc51c2cb86341ed9291}{R script} was executed that retrieved and processed combined the water quality and weather data for each site to esimate metabolism.  For brevity, only two sites at each reserve were evaluated.  Mean annual values at each site, organized by region, are shown in \Cref{fig:metab_plo}, whereas decadal comparisons are shown in \Cref{tab:metab_tab}.  All sites were generally net heterotrophic across the range of observations (i.e., sink of organic matter, in agreement with Caffrey \cite{Caffrey03}), although differences were observed in early (i.e., 1995-2004) as compared to recent (2005-2014) time periods. Overall, the results indicate that between-region and within-site differences in metabolism are apparent and varying by time period, such that a more comprehensive evaluation of factors that influence metabolic rates is needed.  More importantly, the use of the data retreival, synthesis, and analysis functions to create the results shown in \Cref{fig:metab_plo,tab:metab_tab} illustrates the utility provided by the SWMPr package.

% metabolism figure
\begin{knitrout}
\definecolor{shadecolor}{rgb}{0.969, 0.969, 0.969}\color{fgcolor}\begin{figure}[!ht]


{\centering \includegraphics[width=\textwidth]{figure/metab_plo} 

}

\caption[Aggregated estimates of net metabolism, gross production, and total respiration for two sites at each \ac{NERRS} reserve]{Aggregated estimates of net metabolism, gross production, and total respiration for two sites at each \ac{NERRS} reserve.  Values are daily integrated estimates as mean annual values averaged across all years with 95\% confidence intervals.  Two sites were chosen from each reserve that had the longest available time series. Regions are approximations of geographic coordinates and are not exact, e.g., Old Woman Creek (Lake Erie) is Atlantic.\label{fig:metab_plo}}
\end{figure}


\end{knitrout}

% table summary of trends
%latex.default(metabs_tab[, !names(metabs_tab) %in% c("site",     "region")], file = "", caption = cap, cgroup = c("NEM\\textsuperscript{{\\it a}}",     "Pg", "Rt"), n.cgroup = c(2, 2, 2), colheads = rep(c("1995-2004",     "2005-2014"), 3), rowlabel = "Site", rowname = metabs_tab$site,     digits = 2, insert.bottom = foots, rgroup = unique(metabs_tab$region),     n.rgroup = nrgroups, label = "tab:metab_tab", size = "footnotesize")%
\begin{table}[!tbp]
{\footnotesize
\caption{Trends in metabolism for two sites at each of the \ac{NERRS} reserves.  Values are averages of mean annual estimates for each period of observation (1994-2004 and 2005-2014).  Bold values indicate an increase from the first period, whereas italic values indicate a decrease.  Regions are approximations of geographic coordinates and are not exact, e.g., Old Woman Creek (Lake Erie) is Atlantic.\label{tab:metab_tab}} 
\begin{center}
\begin{tabular}{lllcllcll}
\hline\hline
\multicolumn{1}{l}{\bfseries Site}&\multicolumn{2}{c}{\bfseries NEM\textsuperscript{{\it a}}}&\multicolumn{1}{c}{\bfseries }&\multicolumn{2}{c}{\bfseries Pg}&\multicolumn{1}{c}{\bfseries }&\multicolumn{2}{c}{\bfseries Rt}\tabularnewline
\cline{2-3} \cline{5-6} \cline{8-9}
\multicolumn{1}{l}{}&\multicolumn{1}{c}{1995-2004}&\multicolumn{1}{c}{2005-2014}&\multicolumn{1}{c}{}&\multicolumn{1}{c}{1995-2004}&\multicolumn{1}{c}{2005-2014}&\multicolumn{1}{c}{}&\multicolumn{1}{c}{1995-2004}&\multicolumn{1}{c}{2005-2014}\tabularnewline
\hline
{\bfseries Atlantic}&&&&&&&&\tabularnewline
~~acebb&-0.04&{\bf -0.03}&&5.11&{\it 3.81}&&-5.15&{\bf -3.82}\tabularnewline
~~acesp&-0.26&{\bf -0.1}&&4.68&{\bf 4.93}&&-4.94&{\it -5.03}\tabularnewline
~~cbmip&-0.14&{\it -0.17}&&1.02&{\bf 1.91}&&-1.16&{\it -2.08}\tabularnewline
~~cbmmc&-0.43&{\bf -0.24}&&6.06&{\bf 10.13}&&-6.48&{\it -10.38}\tabularnewline
~~cbvgi&0.13&{\it 0}&&6.56&{\it 4.47}&&-6.43&{\bf -4.46}\tabularnewline
~~cbvtc&-0.15&{\bf -0.07}&&5.58&{\bf 6.07}&&-5.73&{\it -6.14}\tabularnewline
~~delbl&-0.59&{\bf -0.16}&&6.68&{\it 6.51}&&-7.27&{\bf -6.67}\tabularnewline
~~delsl&-0.37&{\bf -0.33}&&2.84&{\bf 3.22}&&-3.21&{\it -3.55}\tabularnewline
~~grblr&&-0.02&&&1.28&&&-1.31\tabularnewline
~~grbor&&-0.04&&&3.16&&&-3.2\tabularnewline
~~gtmfm&-0.09&{\bf -0.04}&&3.36&{\bf 4.19}&&-3.45&{\it -4.23}\tabularnewline
~~gtmpc&-0.48&{\bf -0.45}&&1.95&{\bf 2.91}&&-2.43&{\it -3.37}\tabularnewline
~~hudsk&0.02&{\it 0.02}&&-0.36&{\it -0.39}&&0.38&{\bf 0.41}\tabularnewline
~~hudtn&-0.03&{\it -0.03}&&3.75&{\it 3.43}&&-3.78&{\bf -3.46}\tabularnewline
~~jacb6&0.05&{\it 0.03}&&3.41&{\bf 3.93}&&-3.36&{\it -3.89}\tabularnewline
~~jacb9&0.01&{\it -0.02}&&1.8&{\bf 3.1}&&-1.79&{\it -3.12}\tabularnewline
~~narnc&-0.64&{\bf -0.53}&&12.64&{\it 11.33}&&-13.25&{\bf -11.86}\tabularnewline
~~narpc&-0.14&{\bf -0.03}&&7.36&{\it 6.3}&&-7.5&{\bf -6.32}\tabularnewline
~~niwcb&-0.46&{\bf -0.28}&&5.5&{\bf 6.1}&&-5.95&{\it -6.38}\tabularnewline
~~niwdc&-0.13&{\it -0.2}&&7.28&{\bf 8.02}&&-7.41&{\it -8.23}\tabularnewline
~~nocec&-0.05&{\it -0.06}&&4.86&{\bf 4.88}&&-4.92&{\it -4.94}\tabularnewline
~~nocrc&-0.19&{\it -0.19}&&5.93&{\it 5.31}&&-6.12&{\bf -5.5}\tabularnewline
~~owcbr&-0.17&{\bf -0.09}&&1.45&{\bf 1.9}&&-1.62&{\it -1.99}\tabularnewline
~~owcwm&-0.36&{\bf -0.19}&&7.02&{\it 5.36}&&-7.38&{\bf -5.54}\tabularnewline
~~saphd&-1.28&{\bf -1}&&1.89&{\bf 3.77}&&-3.17&{\it -4.77}\tabularnewline
~~sapld&-0.16&{\it -0.43}&&2.32&{\bf 3.56}&&-2.49&{\it -3.99}\tabularnewline
~~welht&-0.07&{\bf -0.04}&&0.36&{\bf 0.63}&&-0.43&{\it -0.67}\tabularnewline
~~welin&-1.87&{\it -2.49}&&3.61&{\it 2.68}&&-5.48&{\bf -5.18}\tabularnewline
~~wqbcr&-0.01&{\bf 0.01}&&8.13&{\it 7.41}&&-8.14&{\bf -7.4}\tabularnewline
~~wqbmh&0.02&{\it 0}&&2.82&{\it 2.76}&&-2.79&{\bf -2.75}\tabularnewline
\hline
{\bfseries Gulf}&&&&&&&&\tabularnewline
~~apaeb&-0.35&{\it -0.35}&&4.19&{\bf 4.31}&&-4.54&{\it -4.67}\tabularnewline
~~apaes&-0.71&{\bf -0.67}&&3.35&{\it 3.05}&&-4.06&{\bf -3.73}\tabularnewline
~~gndbc&-1.02&{\bf -0.88}&&3.5&{\bf 4.3}&&-4.51&{\it -5.18}\tabularnewline
~~gndbh&-1.81&{\it -2}&&2.19&{\it 1.77}&&-4&{\bf -3.77}\tabularnewline
~~job09&-0.34&{\bf -0.25}&&5.32&{\bf 5.95}&&-5.66&{\it -6.2}\tabularnewline
~~job10&-0.15&{\it -0.15}&&3.58&{\bf 4.03}&&-3.72&{\it -4.2}\tabularnewline
~~marab&&-0.02&&&3.38&&&-3.4\tabularnewline
~~marce&&-0.01&&&3.62&&&-3.63\tabularnewline
~~rkbfb&-0.28&{\it -0.37}&&5.28&{\bf 6.25}&&-5.56&{\it -6.62}\tabularnewline
~~rkbfu&-0.33&{\it -0.41}&&2.95&{\bf 3.24}&&-3.28&{\it -3.64}\tabularnewline
~~wkbfr&-0.29&{\bf -0.14}&&6.27&{\bf 6.35}&&-6.57&{\bf -6.49}\tabularnewline
~~wkbwb&-0.29&{\bf -0.16}&&6.81&{\it 6.54}&&-7.1&{\bf -6.69}\tabularnewline
\hline
{\bfseries Pacific}&&&&&&&&\tabularnewline
~~elkap&0.03&{\bf 0.11}&&14.7&{\bf 19.84}&&-14.67&{\it -19.74}\tabularnewline
~~elknm&-0.24&{\bf -0.08}&&11.95&{\bf 16.37}&&-12.18&{\it -16.45}\tabularnewline
~~kachd&0.2&{\it -0.66}&&7.31&{\it 3.2}&&-7.12&{\bf -3.86}\tabularnewline
~~kacsd&-0.06&{\it -0.26}&&5.15&{\it 4.2}&&-5.21&{\bf -4.45}\tabularnewline
~~pdbbp&0.01&{\bf 0.08}&&6.29&{\bf 7.31}&&-6.27&{\it -7.23}\tabularnewline
~~pdbby&-0.07&{\bf -0.01}&&8.39&{\it 5.92}&&-8.47&{\bf -5.93}\tabularnewline
~~sfbcc&&-0.24&&&3.56&&&-3.79\tabularnewline
~~sfbgc&&-0.55&&&7.32&&&-7.87\tabularnewline
~~sosva&-0.2&{\bf -0.05}&&6.98&{\bf 7.83}&&-7.18&{\it -7.87}\tabularnewline
~~soswi&-0.28&{\bf -0.08}&&4.41&{\bf 4.74}&&-4.69&{\it -4.82}\tabularnewline
~~tjrbr&-1.33&{\bf -0.17}&&7.72&{\it 7.06}&&-9.05&{\bf -7.22}\tabularnewline
~~tjros&-0.26&{\it -0.34}&&10.18&{\bf 11.61}&&-10.44&{\it -11.96}\tabularnewline
\hline
\end{tabular}\end{center}}

\footnotesize{\textsuperscript{{\it a}}NEM: net ecosystem metabolism, Pg: gross production, Rt: total respiration, all values in g O$_2$ m$^{-2}$ d$^{-1}$ as annual averages.}\end{table}


The second and third examples are two Shiny web applications \cite{Chang15} that illustrate summaries and comparisons of \ac{SWMP} data within and between reserves that were created using the SWMPr package (see the \nameref{supp_info}).  The second application evaluates trends in \ac{SWMP} data within and between sites using an interactive map.  Trends between reserves can be viewed using the map, whereas trends at individual sites can be viewed by clicking on a map location.  Site-levels trends are described below the map using a simple linear regression to show an increase or decrease in values over time.  Trends on the map at each station are plotted as circles that identify the direction and significance of the trend.  The trend direction is blue for decreasing and red for increasing.  The significance is indicated by radius of the circle and color shading where larger points with darker colors indicate a strong trend.  The third application provides graphical summaries of water quality, weather, or nutrient station data at individual stations. The drop down menus can be used to select the station, date range, and parameter for plotting. Plots are based on daily averages for each parameter.  The data used for each application are similar to those used to estimate ecosystem metabolism described above.  

\section*{Summary}

Add some stuff....

\section*{Supporting Information}
\label{supp_info}

% Include only the SI item label in the subsection heading. Use the \nameref{label} command to cite SI items in the text.
\subsection*{Trends in \ac{SWMP} parameters}
\label{swmp_trends}
Evaluate trends within and between reserves using an interactive map: \href{https://beckmw.shinyapps.io/swmp_comp}{https://beckmw.shinyapps.io/swmp\_comp}

\subsection*{Monthly and annual summary of \ac{SWMP} parameters}
\label{swmp_summary}
Summarize trends in \ac{SWMP} parameters for an individual site: \href{https://beckmw.shinyapps.io/swmp_summary/}{https://beckmw.shinyapps.io/swmp\_summary/} 

\subsection*{\ac{SWMP} data as R binary files}

Quickly access all \ac{SWMP} data from 1995 through 2014. See the help file for \texttt{import\_remote} for more information: \href{https://s3.amazonaws.com/swmpalldata/}{https://s3.amazonaws.com/swmpalldata/}

\section*{Acknowledgments}

I acknowledge the significant efforts of \ac{NERRS} researchers and staff for providing access to high-quality monitoring data.  Thanks particularly to Dwayne Porter and Melissa Ide from \ac{CDMO} for maintaining the online database.  Thanks to Marie Bundy and Nikki Dix for providing me the opportunity to share this package with the broader \ac{NERRS} community.  Thanks to Todd O'Brien for the inspiration for the online widgets in the supporting information. Thanks to Mike Murrell and Jim Hagy III for assistance with documentation and implementation of the ecosystem metabolism functions. The views expressed in this article are those of the authors and do not necessarily reflect the views or policies of the U.S. Environmental Protection Agency.  The use of trade names or products does not constitute endorsement by the US Government.   

\nolinenumbers

\bibliography{swmpr_refs}

\end{document}
